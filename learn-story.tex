\section{\LaTeX 学习历程}

我从2010年开始接触到\LaTeX ,当时忘记了在哪个地方偶然看到了关于\LaTeX 的一些介绍文章,于是在自己的台式机上下载了\TeX Live,跟着指导一点一点学,但没有学习太深入。2011年上了研究生,我重新拾起了对\LaTeX 的兴趣,开始继续深入的学习,记得当时还去北邮的打印店把整本书都印了出来,并且在数值计算的结课上使用了一个期刊的\LaTeX 模板,老师还高高兴兴地给了我高分。现在想起来,大概是学数学的经常使用\LaTeX ,而其他专业的少人问津,当时的数学老师可能比较惊讶。我记得当时没有使用fontspec和XeCJK库,而是使用了MikTeX包和WinEdt编辑器。到了研究生二年级的时候,我接触了Beamer和XeCJK,并开始应用于幻灯片的制作。但是当时缺少系统和深入的学习,幻灯片制作得并不优雅美观。

我使用\LaTeX 最大的成就,就是用它完成了自己的硕士学位论文。在写学位论文时,我查找了大量的资料,后来使用了北邮一个毕业生的模板,做了稍许的修改。现在看来,\LaTeX 排出来的文章,确实在美观性上远好于Microsoft Word的效果。

工作以后的2014-2015年间,也断断续续地学了一些TikZ,也做了一些幻灯片,同时也全面转向了\CTeX 宏包。但是,由于做文档与单位的风格不同,做幻灯片也不够新颖美观,因此慢慢地少于使用。


我学习\LaTeX 完全出于兴趣,而非学习和工作的需要,并且由于以上断断续续的学习,目前我大概还处于初级使用者的水平,远谈不上深入,而且我预计以后的工作中,也会越来越少地使用\LaTeX ,但作为一项基本的书写技能,现在想把以前零碎的笔记做一个系统化的整理,以方便自己以后的使用,同时也希望有利于其他人的学习。