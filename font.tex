\section{字体}
\subsection{涉及字体的几个文件}

以下几个文件涉及字体配置,可参考\url{http://liam0205.me/2016/12/11/LaTeX-traditional-font-scheme/}\footnote{译自\url{http://tex.stackexchange.com/a/119501/38350}}:
\begin{itemize}
	\item .def文件
	\item .fd文件 
	\item .vf文件
	\item .tfm文件
	\item .map文件
	\item .pfb文件
\end{itemize}

使用命令\verb|\sffamily|可将当前字符的字体切换为无衬线字体。

早就有人写了完整的LaTeX字体巡礼:\url{http://www.tug.dk/FontCatalogue/ },这个网站罗列了156个LaTeX中可以免费使用的字体,并且给出了例子和调用的源代码,需要注意的是这些字体并非默认安装在机器上,但至少都能从CTAN得到——不光是宏包,还有字体文件(因为像winfonts,MinionPro这些宏包需要用户自己拥有相应的字体,CTAN上并没有)。 

推荐一下几个字体/字体包:
\begin{description}
	\item[Palatino] Will Robertson的文档总是用Palatino,这字体的名气也不小。胖胖的很活泼,笔锋也优雅,有羽毛笔的进化痕迹。LaTeX中最省事的是用 \verb|\usepackage{mathpazo}| 来统一修改正文和数学字体,这个宏包还有 \verb|[sc, osf]| 参数,分别对应小大写字母和不齐线数字。此外还有一个palatinox宏包可以直接调用Windows系统中的Palatino Linotype(这是微软认证发布赫尔曼·察普夫的原作),相关网址是:\url{http://www.ctan.org/tex-archive/fonts/truetypemetrics/palatinox/},需要手动安装。在这个URL的上一层还能看到另一个经典字体frutiger,只是我手头没有Linotype Frutiger。 
	\item[Garamond] 1530年诞生的经典字体,LaTeX中通过mathdesign可以使用: \verb|\usepackage[garamond]{mathdesign}| 来使用。Garamond字体十分大气,打印在纸张上也特别好看,法国很多口袋图书用的是Garamond。 需要注意的是虽说免费,URW的garamond字体在默认安装的发行版中可能不存在,但是可以下载到,例如\url{http://ctan.binkerton.com/nonfree/fonts/urw/garamond/ }下载所有pfb文件:ugmr8a.pfb ugmri8a.pfb ugmm8a.pfb ugmmi8a.pfb 放到 \verb|font\type1\| 里面的某个目录后刷新数据库即可 
	\item[Times] 除了 \verb|\usepackage{times}| 外,\verb|\usepackage{mathptmx}| 可以把数学字体也改成类似Times的字体。这个字体真的不需要再多说什么了,总之我觉得看久了眼睛会累,但是打印的效果非常稳妥。
	\item[Utopia] Utopia有点像Times,但更宽敞一些。\verb|\usepackage{fourier}| 统一修改正文和数学字体为Utopia, \verb|\usepackage[adobe-utopia]{mathdesign}| 则是mathdesign的调用方法,差别不太明显。
	\item[Avant Garde/Courier/Bookman/New Century Schoolbook] 
不是我懒,这几个字体在PSNFSS中是可以搭配着用的:\verb|\usepackage{avant}| 只载入Avant Garde,\verb|\usepackage{bookman}| 则同时载入Bookman(衬线),Avant Garde(无衬线)和Courier(等宽)字体,\verb|\usepackage{newcent}| 同时载入New Century Schoolbook(衬线),Avant Garde(无衬线)和Courier(等宽)字体 
	\item[Charter] 十分饱满的衬线字体,适合屏幕阅读。\verb|\usepackage{charter}|
	\item[Helvetica/Optima] 这两个字体放一块是因为我觉得它们是无衬线字体,比较适合用来作幻灯片。Helvetica可以 \verb|\usepackage{helvet}| , Optima没有写成宏包的形式,就可以用 \verb|\renewcommand{\sfdefault}{uop}| ,然后 \verb|\renewcommand*\familydefault{\sfdefault}| 来调用。在幻灯片这样的尺寸上,Optima变化的线宽才显现出优美来。 (不过beamer的作者认为Optima不适合做幻灯片)
	\item[其他] 建议看看\url{ftp://tug.ctan.org/pub/tex-archi … t_Survey/survey.pdf} 这篇文章,介绍得相当详细,而且有效果图展示。 
	\item[Minion Pro] \url{http://tug.ctan.org/tex-archive/fonts/minionpro/} 有详细的安装说明,只要不出错是肯定能安上的,装了Acrobat Reader 7.0以上的用户都能在Acrobat安装目录下找到 MinionPro-Bold.otf , MinionPro-BoldIT.otf , MinionPro-It.otf , MinionPro-Regular.otf 这四个文件,按照安装说明拆解它们四个已经能满足日常文档的需要。此外,MnSymbol宏包(MiKTeX可以自动安装)是配合Minion Pro的数学宏包,最好装上,不过 \verb|\usepackage{MinionPro}| 就够了,会自动载入MnSymbol宏包。
\end{description}

其实用来用去才发现,LaTeX自带的这些字体才是真正经过时间和实践检验的经典字体,是TUG智慧的结晶。而且,这150多种字体也涵盖了绝大部分(LaTeX能触及到的)字体使用领域。这是不应该被遗忘的宝藏。

在linux下,可安装\verb|poppler-utils|程序并采用\verb|pdffonts dox.pdf|命令来查看PDF文件中的字体。