\section{关于\TeX 作者Knuth的一些轶事}

传说 Knuth 写书写文章的第一稿都是用铅笔写的。 很多人不明白他为什么不用键盘。 其实原因是这样,Knuth 曾经参加过一个训练小秘的学习班,练习打字每分钟 80 个词以上。到了后来,他发现他打字的速度大大高于他思考的速度,所以如果他用键盘,就会出现很多停顿。所以他决定用铅笔,这样可以与读者的思考速度保持一致。 

Knuth 作为一个计算机科学家,为什么放下他所有的工作10年, 专心研究排版美学,创造 \TeX 系统。这是很奇怪的一件事情。其实原因是这样。真正完美的数学排版应该是用金属活字进行的。但是自从70年代以来,真正懂得这项技术的人都死光了。 新的排版机器,很不幸的都被计算机操纵了 (想想 Matrix ) 虽然当时计算机能够排出一些简单的报纸,杂志, 但是它们不能很好的处理数学公式。Knuth 想写出一个小玩艺儿能够在不同的计算机上制造漂亮的数学公式,于是 \TeX (读作 Tech(nology) 的前半部分) 就诞生了。 

很多人都对 \TeX 断行的算法感到满意,其实只有 Knuth 觉得担心。 他设计 \TeX 的时候听说有一本书叫做 Aesthetic Measures,作者是美国 No.1 数学家 George David Birkhoff。 是说怎样用数学公式来衡量“美”。 他查阅了7个Harvard图书馆,其中有一个图书馆有一个拷贝, 但是却被人借走了。无奈,跑到 MIT 去借。还好,借到了。后来他就在 \TeX 里加入了一个变量叫做 badness,用来衡量一行文字的美感。badness 越小这行文字就越美。但是与 Birkhoff 不同,Knuth 对这个公式没有多少信心。也许是因为谦虚。 

Knuth 的书都是自己用 TeX 排版的,但是却不都是自己设计的。传说 Knuth 和 Graham, Patashnik 合作写 Concrete Mathmatics 的时候请了一位有名的图书版面设计家为他们设计好了书的尺寸,字体大小,标题样式, 后来 Hermann Zapf 专门设计了一种数学字体叫做 Euler, 自此,数学家 Euler 的灵魂浮游于 CM 当中…… 另外一个图书设计家告诉 Knuth 一种格式数学公式的办法, 就是不把数学居中,而是只相对正文缩进一定距离。 


大家都知道 1974 年图灵奖授予 Knuth 主要是因为他写了一部巨著叫做 The Art of Computer Programming 但是不幸的是,很多人不能理解,甚至不相信他为这部书起了这么一个不“科学”的名字。 后来很多人的著作里出现这样的文献引用:"The Act of Computer Programming, Donald Knuth." 

Knuth 是个喜欢自夸的人,这是毫无疑问的。 在他出版 The Art of Computer Programming 之前就已经有这种苗头了。 还没有出版的时候,在一次会议上,有个人知道他的这种性格,就说:“我猜你正在写的这本书的书名肯定是 ‘An Introduction to Don Knuth’。” Knuth 回答说:“正好相反。我要以你的名字来命名它。” 原来这个人的名字叫 "Art Evans". 

Knuth 是Caltech数学系博士毕业的 但是他常常说:“我戴着一顶计算机科学家的帽子,而不是一顶数学家的帽子。” 这说明他似乎对数学家有某些看法。在他看来数学家只知道“What is it”,而他还知道 "How to do it". 这就是他认为的数学与计算机科学的区别。

Knuth 回到 Stanford 时,学校让他自己给自己一个头衔,他就选了一个 Professor Emeritus of The Art of Computer Programming 他其实觉得“计算机科学”不是科学。 虽然大家很希望计算机编程变成科学,这是某ACM刊物提出的忠旨。但是 Knuth 觉得奇怪为什么大家这么喜欢科学,以致于他们瞬间把程序设计变成了科学,方法就是叫它“计算机科学”。Just call it "Computer Science" 在他眼里,计算机科学其实仍然是门艺术。 在 Knuth 的眼里,科学与艺术有什么区别呢? 艺术是人创造的,而科学不是。 艺术永远是可以无止境的提高的,而科学不是。 艺术需要天赋才能掌握,而科学不需要,按部就班就行。所以,The ... ART ... of Computer Programming! 



Knuth 的 The Art of Computer Programming(TAOCP),开始写的也不那么好。 传说有一天 Bob Floyd 给 Knuth 一封信,开门见山就说: “Don, 请不要用那么多感叹号!”信的结尾至少打了五个概叹号。 看了之后,Knuth 发现 TAOCP 里竟然平均每页有两个感叹号!! 

有人说 Knuth 写完三卷 taocp 就去研究 TeX,其实是因为害怕写第四卷。很多人早就希望他放下 TeX,继续写书。 Knuth 说:“一个人要把事情做的完美,只有当他跟上帝的意图保持和谐,现在上帝要我去写第四卷了。” 

Knuth 很推崇随机算法。他批改作业时,一般都是翻到随机一页,仔细看那一页, 之后就对学生的作业有了一个概貌,其它的部分就看的不那么仔细了。Knuth 看书的时候首先看第316页,如果书很短就看第100页。仔细看那一页。之后他就可以说那本书好不好。 据说这样做出判断的正确率很高。不知道是否有很多人跟他学,看316和100. 以后写书要注意把第316页或者100页写好呀! 

你们知道 Knuth 发明了一种程序设计方法叫做 Literate Programming (文学编程) ,把程序当成文学作品来写。这样可以创造永恒的作品, 甚至几十年后还有人用它作为茶余饭后的读物。 他为什么要发明这个东西。原因有2: 1. 他想让一个程序员(也许是他自己)在某一天拿到普利策奖。 2. 他想让提出“Structured Programming”(结构化程序)的那些家伙 在写“非文学程序”的时候,就像他当年写“非结构化程序” 的时候一样觉得自己有罪。 



他的“文学编程”思想最早是在英国 Computer Journal 发表的。 当人问他为什么不在美国发表。他说,美国人没文化,他们不能理解这个东西。 






Knuth 喜欢在他的作品里用 "we" 作为主语, 虽然很多时候文章是他一个人写的。 有人认为使用被动语态好。但是 Knuth 认为不应该大量使用被动语态。 “用 We 可以减少被动语态带来的麻烦。‘we’是指你和你的读者。” 那么怎么称呼作者?答案是: the authors, the first author, 或者直接用名字。 但是他确实反对使用 "I",除非你是身名显赫, 人人尊敬的君王式人物,否则最好不要在论文里用 "I"。 在你描述你的程序时,喜欢说 "we insert the element in the heap" 还是 "it inserts the element in the heap" ? Knuth 总是喜欢用 "we"。显然他已经融于算法的动作之中了。 

虽然他不喜欢论文里用 "I", 但是他喜欢让他的程序自称 "I". 看这里: 


\begin{verbatim}
This is TeX, Version 3.14159 (Web2C 7.3.7x) 

! I can't find file `kkkk.tex'. 

<*> kkkk.tex 



Please type another input file name: 

\end{verbatim}





有很多人跟着他学,把这种称呼顽皮的发挥到极致: 
\begin{verbatim}
Welcome to Scheme 48 0.57 (made by wy on 日 11月 24 13:20:27 CST 2002). 

Copyright (c) 1993-2001 by Richard Kelsey and Jonathan Rees. 

Please report bugs to [email]scheme-48-bugs@martigny.ai.mit.edu[/email]. 

Type ,? (comma question-mark) for help. 

> (define (sq x) (* x 

x)) 



; no values returned 

> 

Exit Scheme 48 (y/n)? <按 Ctrl-D> 

I'll only ask another 100 times. 

Exit Scheme 48 (y/n)? 

I'll only ask another 99 times. 

Exit Scheme 48 (y/n)? 

I'll only ask another 98 times. 

Exit Scheme 48 (y/n)? 

I'll only ask another 97 times. 

Exit Scheme 48 (y/n)? 

I'll only ask another 96 times. 

Exit Scheme 48 (y/n)? 

I'll only ask another 95 times. 

Exit Scheme 48 (y/n)? 

I'll only ask another 94 times. 

Exit Scheme 48 (y/n)? 

...... 

\end{verbatim}



Knuth 有一次布置了一个作业,要求在两个星期以内做出一个用于控制时代广场那种 8x256 像素的阵列显示屏幕的系统, 并且写出一个用户手册。这个用户手册必须让不懂电脑的人也能看懂。 作业交上来以后 Knuth 把文档拿给他夫人(Jill Knuth)看, 结果发现 Jill 对文档里的 "Menu" 和 "Scrolling" 这样的单词都摸不着头脑, 更不要说 "left-indented", "exit", ... 



Knuth 曾经在 American Mathematical Monthly 发表过一篇叫做 "The Toilet Paper Problem" 的论文, 据说是一个研究怎样合理使用厕纸的算法。 现在收录于 Selected Papers on Analysis of Algorithms, p.111 。这篇论文投到 Monthly 时,在节标题使用了大量“粪便学”词汇。 编辑警告 Knuth 说,笑话在我们这里是危险的,请你三思! 不得已啊,Knuth 后来把小节标题里的那种词改掉了。 可是他很不想改文章的标题。 怎么办呢? 他给编辑一封信说,我已经以这个标题做过两次演讲, 这个主题已经被广泛的采用和讨论……云云。 最后编辑回信说:“你的厕纸被接受了!” 


Stanford 计算机科学系楼里的厕纸架上可以并排放两筒厕纸。 人上厕所时有可能从两个中的一个取纸。 两个卷筒不一样大的时候,喜欢从大的那筒拿纸的人叫做 big-chooser 喜欢从小的那筒拿纸的人叫做 little-chooser。 如果两筒大小差不多,一般人都会从离自己最近的那筒取纸。 厕纸用完的时候一般由janitor(门口老大爷?)提供新的纸。 如果一边的用完了,那就换掉那一边,如果两边同时用完,那么有人就有麻烦了……Knuth 似乎在计算那种两筒同时用完的窘境出现的概率…… 论文原文的一个拷贝在: \url{http://learn.tsinghua.edu.cn/homepage/015450/doc/toiletPaperProblem.pdf}比较复杂的数学…… 有兴趣的可以看看。 另外,Don Norman 在这里有一些新设计的厕纸筒可以避免这种情况发生: \url{http://www.jnd.org/dn.mss/ToiletPaperAlgorithms.html}看来 Knuth 是在白费工夫。应该授予 IgNobel 或者 IgFields 


Knuth 和 Graham 他们合写的 Concrete Mathematics 本来不会做的那么花哨的。结果后来他们专门为那本书设计了字体,页面样式, 所有习题都给出了出处。几乎所有页面都至少有一个涂鸦, 连《爱利丝漫游奇境记》都被列入参考文献…… 这是怎么回事呢?原因就是 Knuth 在写书期间去看了一部电影: “白雪公主与七个小矮人”。 看完之后 Knuth 感叹道: 难以置信,这样完美的艺术品竟然可以在1937年完成。 我一定要把我的书做成一个艺术品,而且要在三个月之内完成!结果他说到做到了。