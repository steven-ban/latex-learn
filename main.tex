\documentclass[11pt]{ctexart}


\title{\heiti{\LaTeX 总结}}
\author{\kaishu{北方以北}}
% \institute{国家知识产权局专利局专利审查协作河南中心}
\begin{document}

\maketitle

\tableofcontents

\section{序}

这是我关于自己在\LaTeX 学习和使用中笔记的整理和归纳,其中大部分来自于几年间在为知笔记中的收藏记录,少部分来自于其他人的一些课件或文章。这些内容都是零散的,我只是分门别类做了整理,因此如果想系统学习\LaTeX ,最好还是看官方的教程或书籍。

\section{\LaTeX 学习历程}

我从2010年开始接触到\LaTeX ,当时忘记了在哪个地方偶然看到了关于\LaTeX 的一些介绍文章,于是在自己的台式机上下载了\TeX Live,跟着指导一点一点学,但没有学习太深入。2011年上了研究生,重新拾起了对\LaTeX 的兴趣,开始继续深入的学习,记得当时还去北邮的打印店把整本书都印了出来,并且在数值计算的结课上使用了一个期刊的\LaTeX 模板,老师还高高兴兴地给了我高分。现在想起来,大概是学数学的经常使用\LaTeX ,而其他专业的少人问津,当时的数学老师可能比较惊讶。我记得当时没有使用fontspec和XeCJK库,而是使用了MikTeX包和WinEdt编辑器。到了研究生二年级的时候,我接触了Beamer和XeCJK,并开始应用于幻灯片的制作。但是当时缺少系统和深入的学习,幻灯片制作得并不优雅美观。工作以后的2014-2015年间,也断断续续地学了一些TikZ,也做了一些幻灯片,同时也全面转向了\CTeX 宏包。但是,由于做文档与单位的风格不同,做幻灯片也不够新颖美观,因此慢慢地少于使用。目前我大概还处于初级使用者的水平,远谈不上深入,而且我预计以后的工作中,也会越来越少地使用\LaTeX ,但作为一项基本的书写技能,现在想把以前零碎的笔记做一个系统化的整理,以方便自己以后的使用,同时也希望有利于其他人的学习。

\section{\LaTeX 基本语法拾遗}

\section{\LaTeX 中文排版}

\section{Beamer拾遗}

\section{文档类编写}

\section{总结}

\end{document}