\documentclass[linespread=1.5]{ctexart}

\newfontfamily\code{Courier New}

\usepackage[margin=2cm]{geometry}
\usepackage{mdwlist}

\usepackage{graphicx}

\usepackage{listings}
\lstset{
	numbers=left,
	numberstyle=\tiny\code,
	basicstyle=\small\ttfamily,
	frame=lines,
	keywordstyle=\color{red},
	identifierstyle=,
	commentstyle=color{grey},
	stringstyle=\code,
	showstringspaces=false,
	language=tex
}

\usepackage{url}
\usepackage{hyperref}

\title{\heiti{\LaTeX 总结}}
\author{\kaishu{北方以北}}
% \institute{国家知识产权局专利局专利审查协作河南中心}
\begin{document}

\maketitle

\tableofcontents

\section{序}

这是我关于自己在\LaTeX 学习和使用中笔记的整理和归纳,其中大部分来自于几年间在为知笔记中的收藏记录,少部分来自于其他人的一些课件或文章。这些内容都是零散的,我只是分门别类做了整理,因此如果想系统学习\LaTeX ,最好还是看官方的教程或书籍。

\input{learn-story.tex}

\input{about-Knuth.tex}

\section{\LaTeX 基本语法拾遗}
\LaTeX 是向后兼容的。

\subsection{引用宏包}
nag宏包。事实上,如果你的代码没问题,这个宏包将不会做任何事情。注意:把这个宏包放在你的导言区的第一行(甚至在 \verb|\documentclass| 之前)。它将会检测你文档中是否使用已经被淘汰了的宏包以及过时的命令。

\subsection{排版基础}


在paper开头,必不可少的是会引用LaTeX packages。除了常用的\LaTeX 包,在此需要特别说明的有两个:一是\verb|\usepackage[american]{babel}|,另一个是\verb|\usepackage{microtype}|,引用这两个包会大大提升排版的正式程度和美观程度。其中\verb|\usepackage[american]{babel}|的引用是遇单词换行的时候,确保单词的切割是按照音节来而不是随意切割。这会让作为native speaker的审稿人赏心悦目,心中暗爽。而\verb|\usepackage{microtype}|的最大特点就是能够调整全篇文章(或局部)的字间距,字间距最大调整范围为±1em。可使得某段落不会出现单独一个单词占一行,或文章末尾单独一行文字占一页的不美观情况。\footnote{该包在NIPS中自带引用,而ECCV由于LNCS在排版方面的一些限制因素,不推荐在ECCV中使用该包引用}如果有使用到字体宏包,需要将 microtype 宏包放在它们的后面,因为这个宏包对单词、字母的调整和字体是有关的。

在行文过程中若使用括号,括号前一定要有空格与前文内容分开。这也是中文作者很容易忽略的一点。例如,“the boosting algorithm (AdaBoost)” “the clustering algorithm (k-means)”。

siunitx 宏包大大简化了写作科技文的 \TeX 命令,科技文写作中很大一部分是单位、数字。这个宏包添加了一些命令,比如 \verb|\num|命令可以输出我们想要的各种方式的数字形式(比如科学记数法),而 \verb|\si| 命令用来输出单位。我经常用到的命令是 \verb|\SI| 和 \verb|\SIrange|。比如 \verb|\SI{10}{\hertz}| 输出为 “10Hz”(这能有效避免输入错误,我可能会写成 HZ 或者 hz 而不是 Hz)。\verb|\SIrange| 命令多一个参数:\verb|\SIrange{10}{100}{\hertz}| 输出为 “10Hz to 100Hz”。

hyperref 非常强大,你可以有非常多的可能性,其中最突出的特色是超链接。当引用一幅图的时候,引用与图形形成了链接,当你点击引用的地方,它会跳转到链接的图片处。并且 hyperref 可以让你插入 PDF 元数据到你的最终文档中。注意:作为一个经验法则,你应该在导言区的最后加入这个宏包,在所有宏包之后。也存在少数例外的情况:比如,cleveref宏包应该在 hyperref 之后。

单双引号的使用。正确的单引号使用方式:左引号为 (全键盘上数字1左侧键位),右引号为 ' (分号右侧键位)。双引号则重复输入两次即可(即 '')。切记不是平常引号的输入方式。

脚注使用时如遇句尾,则应该在句尾标点符号后引用,具体为:“respectively.\verb|\footnote{blablabla}”|。这样是为了防止句尾是数字的情形产生误读:如果是数字,footnote的标号就很可能被误认为数字的次方。

一些英文简写的用法。“that is”写作“i.e.,”,“for example”写为“e.g.,”,而“参看/参考”简写为“cf.”。注意,前两者有两个“.”且末尾要有“,”而“参考”的简写只有一个“.”。

文中在引用参考文献时要用“\~{}”(而不是直接空格)来产生空格。例如,“state-of-the-art MIL algorithms, e.g., miFV\~{}\verb|\cite{bibmiFV}| and miGraph\~{}\verb|\cite{bibmiGraph}|, and …” 用“\~{}”来产生空格的好处是使得“miFV [5]”作为一个整体,在换行时不会发生“[5]”与前文分开而单独处于行首的错误情况。“\~{}\verb|\ref{}|”命令同理。

几种分隔方式:
\begin{description*}
	\item[$\backslash \backslash$ \it{offset} ] 换行,并且与下一行的行间距为原来行间距+offset。
	\item[$\backslash$ newline] 与\verb|\\|相同。
	\item[$\backslash$ linebreak] 强制换行,与\verb|\newline|的区别为\verb|\linebreak|的当前行分散对齐。
	\item[$\backslash$ par] 分段。
\end{description*}

\verb|\newpage|分页命令,\verb|\clearpage|和 \verb|\newpage| 类似。

如果想要首行缩进两个汉字距离,则\verb|\setlength{\parindent}{2em}|。

注册商标外面那个R带一个圈可以写为\verb|\textregistered|(注册)或\verb|\texttrademark|(商标),也可以使用\verb|\textcircled{\scriptsize R}|。

为产生目录中的简写形式,可使用\verb|\caption[short text]{long text}|,其中方括号中为简写形式。

段落结尾时,不要使用\verb|\\|,使用双反斜杠是为了本段继续书写。

如果编译时出错,可在出错前的命令前添加\verb|\protect|用于调试,但应修复错误。

\LaTeX 类及宏包是在\verb|\leftmark|和\verb|\rightmark|中存储章节编号的。

为改变输出尺寸,可使用\verb|landscape|宏包。页面布局可使用宏包\verb|typearea|。页面间距可使用\verb|geometry|宏包。



英文中常用到连笔,但有些PDF阅读软件并不支持连笔的搜索和复制,此时可在\verb|microtype|宏包中设置\verb|noligatures|选项来进行取消。

命令\verb|\xspace|可根据不同的情形添加空格:后面跟一个单词时,加入一个空格;如果后面是英文的句号、逗号、叹号和引号则不添加空格。

引入宏包\verb|mhchem|可书写化学式,如\verb|\ce{(NH4}2SO4|。\footnote{参见\url{http://en.wikibooks.org/wiki/File:Ammonium_sulphate_mhchem.png}}

使用\verb|pandoc|可将\TeX 文件转换为各种格式的文档,包括电子书。



\subsection{行间距与段间距}

LaTeX  支持的长度单位有:
\begin{itemize*}
	\item in - 英寸(inch)
	\item mm - 毫米(millimeters)
	\item cm - 厘米(centimeters)
	\item pt - points (大约 $1/72$ inch)
	\item em - 接近当前字体的字符 "M"的宽度(approximately the width of an "M" in the current font)
	\item ex - 接近当前字体的字符 "x"的高度approximately the height of an "x" in the current font
\end{itemize*}
长度也可以是负值, 例如  -1.5em.  注意到数字 0 本身不是一个长度,  必须表明是0in 或 0pt 等.

\LaTeX 默认行间距为行高的20\% ,为调整行间距,可使用\verb|setspace|宏包,其具有选项\verb|onehalfspacing|可将行间距调整为1.5倍行高。宏包\verb|parskip|可用于取消\emph{段落}的缩进,此宏包中还包括段落间的距离调整功能。

段落间距有关变量:
\begin{itemize*}
	\item \verb|\baselineskip|:行基线间距。
	\item \verb|\lineskip|  :行间距。
	\item \verb|\baselinestretch|:伸展因子。
	\item \verb|\parskip|:部分段间距。
	\item \verb|\lineskiplimit|:当两行字之间的距离小于\verb|\lineskiplimit|时,行距自动设为\verb|\lineskip|。
\end{itemize*}
其中段间距等于\verb|\lineskip| + \verb|\parskip|,行间距等于\verb|\lineskip| = \verb|\baselineskip| 乘以 \verb|\baselinestretch|

一般在中文文章中,将 \verb|\parskip| 设置为 0pt,即行间距和段间距相等。设置伸展因子调整行距比较不靠谱,因为经常调不对,索性直接通过 \verb|\fontsize| 直接调 \verb|\baselineskip|,使得 \verb|\baselinestretch| 一直是1,倒来的精确。通过设置伸展因子调整行距不靠谱的原因是默认的 \verb|\baselineskip| 大于字体大小,因此如果你伸展因子设置为 1.5,则实际得出的行距要大于 1.5 倍行距,示例如下:
\begin{lstlisting}
\renewcommand{\baselinestretch}{xx}:设置伸展因子。
\renewcommand{\baselinestretch}{1.5} \normalsize:不改变字体大小,只改变行距。
\end{lstlisting}

通过在源文件中输入 \verb|\showthe\baselineskip|,可以在编译时得到当前的\verb|\baselineskip|的值,并显示出来(在编译时会停住,显示 \verb|\baselineskip| 的值)

\subsection{对齐}
一行对齐:\verb|\leftline{左对齐}| \verb|\centerline{居中}| \verb|\rightline{右对齐}|

多行或者段落对齐:
\begin{itemize}
	\item 左对齐 \verb|\begin{flushleft}...\end{flushleft}|
	\item 居中 \verb|\begin{center}...\end{center}|
	\item 右对齐 \verb|\begin{flushright}...\end{flushright}|
\end{itemize}

\subsection{页眉页脚}
普通的页眉页脚可以使用\verb|\pagestyle{plain}|或\verb|\pagestyle{headings}|,前者不产生页眉页脚,后者产生普通的页眉页脚。

为产生更丰富形式的页眉页脚,可使用fancyhdr宏包来定制,引入宏包后使用\verb|\pagestyle{fancy}|来定制具体的格式。

使用中发现footnote会被分页,可以在某处添加\verb|\interfootnotelinepenalty=10000|来禁止。



\subsection{数学公式排版}
文中,特别是在equation环境下,如果要插入公式,则公式后一定要有标点“逗号”或“句号”。使用方法:在公式后加入“\verb|\,,|”(逗号)或“\verb|\,.|”(句号)即可。不推荐使用\verb|\text{,}|或\verb|\text{.}|。因为\verb|\text{}|环境下的标点长相与“\verb|\,,|”或“\verb|\,.|”不同,且“\verb|\,,|”或“\verb|\,.|”前会自动与公式隔出一段距离,更加正式、美观。

行内高度会压缩的分数\verb|\frac|不太好看,一般用\verb|\dfrac|替换。

公式中的\verb|\ldots|和\verb|\cdots|。“\verb|\ldots|”是列举中的省略符号,而“\verb|\cdots|”用于运算(如连加、连乘等)中的省略,二者主要区别在于位置一高一低,切勿混用。

公式中如果有指数或对数表示,要用\verb|\exp|或\verb|\log|命令。不能用\verb|\text{exp}|或\verb|\text{log}|,更不能直接输入exp或log来表示。

在很多机器学习和视觉文章中会用到范数,正确的一范数或二范数表示应为\verb|$\ell_1$或$\ell_2$|,输出为$\ell_1$或$\ell_2$。

公式中的单位向量或零向量要用向量写法:\verb|\vec{1}| 或\verb|\vec{0}|,有时也用\verb|\bm{1}|加粗来表示向量,否则会被误认为标量。

\verb|\left \right|:可以用.匹配不存在的左右部分 

矩阵分块:在array环境中加线,类似tabular, 

虚线:用arydshln宏包 

用\verb|;{}|代替竖线条 

用\verb|\hdashline|代替代替\verb|\hline|,或下载pmat宏包

图\ref{fig:greeks}是希腊字母的命令表示。
\begin{figure}[htpb]
\centering
\includegraphics[width=0.7\linewidth]{greeks.png}
\caption{希腊字母的命令表示。}\label{fig:greeks}
\end{figure}

使用\verb|\boldmath|命令可切换数学模式下的粗体。宏包bm是专门用来处理数学粗体的,使用\verb|\bm|命令进行切换。然而,上述方式中可能会出现伪粗体,降低了排版的专利性和美观性,因此应尽量选择字重齐全的字体,如newtxmath、stix、pxfonts、mathdesign(只包含符号)、MnSymbol(只包含符号)、fdsymbol(只包含符号)、lucidabr(商业字体)。具体情形参考\url{https://www.zhihu.com/question/25290041}。


\LaTeX 环境下,最重要的数学宏包是AMS系列的。首先是这些主要提供环境的包:
\begin{itemize}
	\item amsmath,最基本的数学包
	\item amscd,绘制交换图
	\item amsthm,制作引理
	\item amsxtra,对于旧式数学式的支持
	\item upref,Roman体的\verb|\ref|输出
\end{itemize}
下面是提供字体的包:
\begin{itemize}
	\item amsfont,定义了\verb|\mathfrak|和\verb|\mathbb|,另有诸多符号
	\item amssymb,另一个定义了诸多符号的包
	\item eufrak,Euler Fraktur
	\item eucal,定义了Euler字体版本的\verb|\mathcal|
\end{itemize}
对于AMS-Math,需要掌握下面最基本的几个环境:
\begin{itemize}
	\item equation equation* 单行单公式
	\item multline multline* 多行公式,没有对齐操作,只给一个公式编号
	\item gather gather* 多个公式,可添加多个公式编号
	\item align align* 多个公式对齐,但只能对齐公式内部的一个部分
	\item flalign flalign* 多个公式对齐,可对公式内的多个部分
	\item split 分割公式
	\item gathered 和gather的区别是放在了一个minipage里
	\item aligned 也是minipage的问题
\end{itemize}
注意:\emph{方程组不要用eqnarray,而要用align},因为前者用两个\verb|&|对齐,对齐时的位置空格偏大,后者用一个\verb|&|对齐,间距比较自然。

数学模式下涉及大量专有符号,可参考The LaTeX Companion或在TexDoc下搜索symbol。


\subsection{列表环境}
类型有三种:itemize,enumerate和description,均基于list环境。

默认的列表环境空白较大,如果需要更紧凑的列表方式,可以选用 mdwlist 宏包提供的 \verb|itemize*|、\verb|enumerate*| 和 \verb|description*| 环境,用法和无星号的版本一致。

在列表的内部,很容易改变一些距离
\begin{lstlisting}
\begin{itemize}
 \setlength{\itemsep}{1pt}
 \setlength{\parskip}{0pt}
 \setlength{\parsep}{0pt}
 \item first item
 \item second item
\end{itemize}
\end{lstlisting}

枚举的列表计数可以通过其计数器来改变。enumerate 提供了四个计数器 enumi,enumii,enumiii, enumiv 对应不同层次的枚举。下面的示例会产生第5个计数:
\begin{lstlisting}
\begin{enumerate}
 \setcounter{enumi}{4}
 \item fifth element
\end{enumerate}
\end{lstlisting}

enumberate宏包为\verb|enumberate|提供了更灵活的标签,如:
\begin{lstlisting}
usepackage{enumerate}

\begin{enumerate}[(i)]
 \item The first item
 \item The second item
 \item The third etc \ldots
\end{enumerate}
\end{lstlisting}

paralist 宏包提供了 \verb|inparaenum| 环境,产生不分段的列表。同样支持跟上面 enumerate 宏包类似的格式化标签。


\subsection{表格排版}
\subsubsection{多列}
表格的多列示例:
\begin{verbatim}
\multirow{number of rows}{width}{entry text}
\multicolumn{number of columns}{formatting options}{entry text}
\end{verbatim}

\subsubsection{表格内文字}

如果想让标题放在表格的上方就把caption加在\verb|\begin{table}|和\verb|\begin{tabular}|之间,如果想把标题放在表格下方,就把caption放在\verb|\end{table}|和\verb|\end{tabular}|之间。

想在表格下方加以段话的话,就在\verb|\end{tabular}|和\verb|\end{table}|直接加一句话,有时候要用\verb|\newline|,有时候不用。但加的这段注释有时候是doublespace的,想变singlespace,加一句\verb|\linespread{1}|就行了。

\subsubsection{如何防止表格过宽?}
\LaTeX 中表格宽度需要手工输入尺寸来控制,往往会越过边界线。如果编译后出现了这些情况,应该如何解决呢?
\begin{itemize}
	\item 使用更小的字体,如\verb|\small{\begin{tabular}...\end{tabular}}|
	\item 手动控制宽度
	\item 使用宏包makecell
	\item 使用宏包tabularx来自动改变列宽
	\item 使用rotating宏包中的sidewaystable宏包
\end{itemize}

\subsubsection{彩色表格}
使用\verb|colortbl|宏包可制作彩色表格,主要使用的命令是\verb|\columncolor|和\verb|rowcolor|,前者给列进行着色,后者给行进行着色。代码如下\footnote{参见\url{http://tex.stackexchange.com/questions/176220/fancy-colored-array-in-latex}}:
\begin{lstlisting}[language=tex]
\documentclass[a4paper,12pt]{article}
\usepackage[left=1.5cm,right=1.5cm,top=1.5cm,bottom=1.5cm,ignoreheadfoot]{geometry}
\usepackage{array}
\usepackage[svgnames,table]{xcolor}
 
\newcommand*{\arraycolor}[1]{\protect\leavevmode\color{#1}}
\newcolumntype{A}{>{\columncolor{blue!50!white}}c}
\newcolumntype{B}{>{\columncolor{LightGoldenrod}}c}
\newcolumntype{C}{>{\columncolor{FireBrick!50}}c}
\newcolumntype{D}{>{\columncolor{Gray!42}}c}
 
\begin{document}    
 
\begin{center}
\sffamily
\arrayrulecolor{white}
\arrayrulewidth=1pt
\renewcommand{\arraystretch}{1.5}
\rowcolors[\hline]{3}{.!50!White}{}
\begin{tabular}{A|B|C}
  \multicolumn{3}{D}{\bfseries Example table}\\
  \rowcolor{.!50!Black}
  \arraycolor{White}\bfseries First column &
  \arraycolor{White}\bfseries Second column&
  \arraycolor{White}\bfseries Third column\\
  1 & A & E\\
  2 & B & F\\
  3 & C & G\\
  4 & D & H\\
\end{tabular}
\end{center}
 
\end{document}
\end{lstlisting}

得到的表格图\ref{fig:colortbl}所示。
\begin{figure}[htpb]
\centering
\includegraphics[width=0.5\linewidth]{colortbl.png}
\caption{彩色表格示例。}\label{fig:colortbl}
\end{figure}

\subsubsection{从外部文件读入表格数据}
有时需要从外部文件中读入大量表格数据,此时手工填写过于繁琐,而在存在外部文件的情况下,是可以直接读取的,方式是使用databool宏包,具体参见\url{http://uweziegenhagen.de/?p=3100}。



\subsection{插图排版}
若文章需要插入很多图片,那么直接和\verb|“.tex”|放在一起就会很杂乱。这时可在\verb|”.tex”|所在目录下新建名为“figure”的文件夹,将图片放入figure文件夹后调用\verb|\graphicspath{{figure/}}|命令即可。

可将\verb|\vspace{-1em}|放置于figure或table的\verb|\caption|和图片或表格主体之间,来缩小空白节省篇幅。当然,如果会议或期刊明确要求了图片主体与图片标题的间距,则不要使用该命令!

插入图片时需要估算本页还有多少空间可用,下面的示例可以根据这个空间来确定图片的高度:
\begin{verbatim}
\includegraphics[
  height=\dimexpr\pagegoal-\pagetotal\relax,
  width=\textwidth,
  keepaspectratio]{graph} 
\end{verbatim}

在 \LaTeX 中,相同文件名加载图片的顺序是:.png .pdf .jpg .mps .jpeg .jbig2 .jb2 .PNG .PDF .JPG .JPEG .JBIG2 .JB2,而这一顺序存储在宏 \verb|\Gin@extensions| 之中。因此,若是在相同目录下同时含有 dummy.png 和 dummy.pdf, 编译引擎将会选择前者(假设支持)。你可以用 \verb|\DeclareGraphicsExtensions| 命令来声明并改变这些扩展名的顺序,比如:
\begin{verbatim}
\DeclareGraphicsExtensions{
    .png,.PNG,%
    .pdf,.PDF,%
    .jpg,.mps,.jpeg,.jbig2,.jb2,.JPG,.JPEG,.JBIG2,.JB2}
\end{verbatim}
这样就能确保 png 文件在 pdf 文件之前被载入了——在最终输出之时只需要交换两行的顺序即可。另外 grfext 宏包提供了 \verb|\PrependGraphicsExtensions| 命令实现同样的效果:
\begin{lstlisting}
\usepackage{grfext}
\PrependGraphicsExtensions*{.png,.PNG}
\end{lstlisting}

在某些情况下,可能需要将多个图并列为一组,而对每幅图均保持独立性,此时可使用\verb|paisubfigure|中的\verb|\subfigure|命令,如下示例:
\begin{lstlisting}
\begin{figure}
   \centering   
	\subfigure[Small Box with a Long Caption]{
     \label{fig:subfig:a} %% label for first subfigure     
	\includegraphics[width=1.0in]{graphic.eps}}   
	\hspace{1in}   
	\subfigure[Big Box]{     
	\label{fig:subfig:b} %% label for second subfigure     
	\includegraphics[width=1.5in]{graphic.eps}}   
	\caption{Two Subfigures}   
\label{fig:subfig} %% label for entire figure \end{figure}
\end{lstlisting}

\subsection{参考文献}
若某文献标题中含有特定含义大写字母(“SVM” “EM”等),应特别用第二重\verb|{}|将其括起来才可使其正常表示。如,“Title = \verb|{{EM-DD}|: An improved multiple-instance learning technique\verb|}|”。

为方便管理和引用参考文献,可采用bibtex,首先是将\verb|.bib|文件放入主文件夹,然后在正文中引入参考文献名称:
\begin{lstlisting}
\bibliographstyle{plain}
\bibliography{filename.bib}
\end{lstlisting}
在需要引入参考文献的地方使用\verb|\cite{cite1}|,其中\verb|cite1|是\verb|filename.bib|文件中的参考文献标识,然后采用如下顺序编译文件:
\begin{lstlisting}
$ xelatex test.tex
$ bibtex test.tex
$ xelatex test.tex
$ xelatex test.tex
\end{lstlisting}
就能得到编译好的文件了。注意:如果一个文件中包含了多个其他的tex文件,在用bibtex进行编译时,应该逐个编译每个被包含的源文件。

一般的期刊数据库中均可以导出bib文件,其格式是固定的。如果bib文件太多,可通过JabRef软件来管理。JabRef可实现全文文献自动关联的功能,即文献字段中有一个\verb|Bib Textkey|可将每篇文献自动生成唯一的域名从而实现自动关联。

在一个文档中多次使用bib文件,如此使用参考文献的情况不多,但是有时也有类似的场合需要这样使用,推荐给各位这一个包:multibib。

\subsection{插入代码}
可以直接使用\verb|\verb||code|来使用latex中的代码,或通过verbatim环境。

\section{字体}
\subsection{涉及字体的几个文件}

以下几个文件涉及字体配置,可参考\url{http://liam0205.me/2016/12/11/LaTeX-traditional-font-scheme/}\footnote{译自\url{http://tex.stackexchange.com/a/119501/38350}}:
\begin{itemize}
	\item .def文件
	\item .fd文件 
	\item .vf文件
	\item .tfm文件
	\item .map文件
	\item .pfb文件
\end{itemize}

使用命令\verb|\sffamily|可将当前字符的字体切换为无衬线字体。

早就有人写了完整的LaTeX字体巡礼:\url{http://www.tug.dk/FontCatalogue/ },这个网站罗列了156个LaTeX中可以免费使用的字体,并且给出了例子和调用的源代码,需要注意的是这些字体并非默认安装在机器上,但至少都能从CTAN得到——不光是宏包,还有字体文件(因为像winfonts,MinionPro这些宏包需要用户自己拥有相应的字体,CTAN上并没有)。 

推荐一下几个字体/字体包:
\begin{description}
	\item[Palatino] Will Robertson的文档总是用Palatino,这字体的名气也不小。胖胖的很活泼,笔锋也优雅,有羽毛笔的进化痕迹。LaTeX中最省事的是用 \verb|\usepackage{mathpazo}| 来统一修改正文和数学字体,这个宏包还有 \verb|[sc, osf]| 参数,分别对应小大写字母和不齐线数字。此外还有一个palatinox宏包可以直接调用Windows系统中的Palatino Linotype(这是微软认证发布赫尔曼·察普夫的原作),相关网址是:\url{http://www.ctan.org/tex-archive/fonts/truetypemetrics/palatinox/},需要手动安装。在这个URL的上一层还能看到另一个经典字体frutiger,只是我手头没有Linotype Frutiger。 
	\item[Garamond] 1530年诞生的经典字体,LaTeX中通过mathdesign可以使用: \verb|\usepackage[garamond]{mathdesign}| 来使用。Garamond字体十分大气,打印在纸张上也特别好看,法国很多口袋图书用的是Garamond。 需要注意的是虽说免费,URW的garamond字体在默认安装的发行版中可能不存在,但是可以下载到,例如\url{http://ctan.binkerton.com/nonfree/fonts/urw/garamond/ }下载所有pfb文件:ugmr8a.pfb ugmri8a.pfb ugmm8a.pfb ugmmi8a.pfb 放到 \verb|font\type1\| 里面的某个目录后刷新数据库即可 
	\item[Times] 除了 \verb|\usepackage{times}| 外,\verb|\usepackage{mathptmx}| 可以把数学字体也改成类似Times的字体。这个字体真的不需要再多说什么了,总之我觉得看久了眼睛会累,但是打印的效果非常稳妥。
	\item[Utopia] Utopia有点像Times,但更宽敞一些。\verb|\usepackage{fourier}| 统一修改正文和数学字体为Utopia, \verb|\usepackage[adobe-utopia]{mathdesign}| 则是mathdesign的调用方法,差别不太明显。
	\item[Avant Garde/Courier/Bookman/New Century Schoolbook] 
不是我懒,这几个字体在PSNFSS中是可以搭配着用的:\verb|\usepackage{avant}| 只载入Avant Garde,\verb|\usepackage{bookman}| 则同时载入Bookman(衬线),Avant Garde(无衬线)和Courier(等宽)字体,\verb|\usepackage{newcent}| 同时载入New Century Schoolbook(衬线),Avant Garde(无衬线)和Courier(等宽)字体 
	\item[Charter] 十分饱满的衬线字体,适合屏幕阅读。\verb|\usepackage{charter}|
	\item[Helvetica/Optima] 这两个字体放一块是因为我觉得它们是无衬线字体,比较适合用来作幻灯片。Helvetica可以 \verb|\usepackage{helvet}| , Optima没有写成宏包的形式,就可以用 \verb|\renewcommand{\sfdefault}{uop}| ,然后 \verb|\renewcommand*\familydefault{\sfdefault}| 来调用。在幻灯片这样的尺寸上,Optima变化的线宽才显现出优美来。 (不过beamer的作者认为Optima不适合做幻灯片)
	\item[其他] 建议看看\url{ftp://tug.ctan.org/pub/tex-archi … t_Survey/survey.pdf} 这篇文章,介绍得相当详细,而且有效果图展示。 
	\item[Minion Pro] \url{http://tug.ctan.org/tex-archive/fonts/minionpro/} 有详细的安装说明,只要不出错是肯定能安上的,装了Acrobat Reader 7.0以上的用户都能在Acrobat安装目录下找到 MinionPro-Bold.otf , MinionPro-BoldIT.otf , MinionPro-It.otf , MinionPro-Regular.otf 这四个文件,按照安装说明拆解它们四个已经能满足日常文档的需要。此外,MnSymbol宏包(MiKTeX可以自动安装)是配合Minion Pro的数学宏包,最好装上,不过 \verb|\usepackage{MinionPro}| 就够了,会自动载入MnSymbol宏包。
\end{description}

其实用来用去才发现,LaTeX自带的这些字体才是真正经过时间和实践检验的经典字体,是TUG智慧的结晶。而且,这150多种字体也涵盖了绝大部分(LaTeX能触及到的)字体使用领域。这是不应该被遗忘的宝藏。

在linux下,可安装\verb|poppler-utils|程序并采用\verb|pdffonts dox.pdf|命令来查看PDF文件中的字体。

\section{\LaTeX 中文排版}
\LaTeX 是老外写的,并未考虑中文的排版。90年代后期以来,陆续出现了CCT等中文排版方式,但由于用的人少同时标准不统一,技术落后,一直没有特别好的解决方式。中文的排版一方面是字体,另一方面是中文对断行、缩进等特殊的排版需求。对前者而言,由于\LaTeX 的字体机制与Microsoft Office不同,需要手动生成相应的字体文件,既不方便,体积又大,让初学者望而却步。后来,刘海洋等人基于Xe\LaTeX 开发完成了CTeX,完美解决了中文排版问题。目前(2017年),\CTeX 已经非常成熟,装好\TeX Live后开箱可用,大大方便了中文排版,以往的技术(包括Xe\LaTeX 与xunicode、fontspec宏包的搭配)统统可以抛弃了。

\CTeX 宏包包括提供的文档类及宏包,其中文档类包括\verb|ctexart.cls|、\verb|ctexrep.cls|和\verb|ctexbook.cls|,分别替代了\LaTeX 标准文档类\verb|article.cls|、\verb|report.cls|和\verb|book.cls|。宏包方面,\verb|ctex.sty|提供该宏包的全部功能,但默认不开启章节标题设置功能,需要使用\verb|heading|选项来开启。\verb|ctex.sty|可在其他文档类(如\verb|beamer|、\verb|moderncv|)中使用以提供完整的中文支持功能。在2017年,还增加了\verb|ctexbeamer.cls|。

\CTeX 中使用\verb|\ctexset{}|来设置相应的选项,具体参见说明文档。

字体类型方面,\CTeX 提供了多个\verb|fontset|选项来支持多种字体族,包括none、adobe、fandol、founder(方正字体)、mac、ubuntu(文泉驿与文鼎)、windows、windowsnew、windowsold,并定义了字体命令\verb|\songti|、\verb|\heiti|、\verb|\fangsong|、\verb|\kaishu|,同时在Windows下还具有\verb|\lishu|和\verb|\youyuan|。ubuntu中没有仿宋字体。在绝大多数情况下,系统中自带的这些字体就可以满足普通的排版需要了,没有必要再使用其他付费或花哨字体。当然,如果想有更好的排版效果,可以考虑以下设计感较强的字体:
\begin{itemize}
	\item 方正正纤黑简体
	\item 汉信全唐诗简体
	\item 微软雅黑\_light
	\item 方正宋刻本秀楷简体
	\item 康熙字典体(繁体)
	\item 欧体楷书
	\item 方正苏新诗柳楷简体
	\item 方正简启体
	\item 长城中行书体
	\item 冬青黑体
	\item 方正兰亭刊宋
	\item 方正清刻本悦宋简体
\end{itemize}
如果需要使用上述默认设置外的字体,需要使用\verb|fontspec|宏包来载入已在系统中安装的这些字体(注意版权问题!),引入宏包后,使用下述命令来设置字体:
\begin{itemize}
	\item \verb|\setmainfont{ }| 一般就是论文中西文部分默认使用的字体。通常到 Word 2003 为止,这里的默认字体都会是 Times New Roman。Linux 下也有同名字体。
	\item \verb|\setsansfont{ }| 是西文默认无衬线字体。一般可能出现在大标题等显眼的位置。这一部分经常碰上的字体会是 Helvetica或Arial。Linux下也有 Helvetica,前缀是 \verb|-adobe-helvetica-*| 。这是一个古老的非抗锯齿版本(也就是不用 fontconfig 配置而使用 xfontsel),所以如今的 XWindow 环境应该不会再使用它作为屏幕字体。而 Windows 下的 Word 则经常会默认为 Arial。Word 和 Windows 本身都不自带 Helvetica。
	\item \verb|\setmonofont{ }|是西文默认的等宽字体。一般用于排版程序代码。Courier 或者 Courier New 是常见的 Word 选项。Linux 下一般会有 Courier,但很少能看见 Courier New。
	\item \verb|\setCJKmainfont[BoldFont={ },ItalicFont={ }]{ }|这个命令指定中文(或韩文日文)的默认字体。通常情况下,我见过的大部分文档论文会要求用宋体排版,实际上就是说的是SimSun。Linux 下可能用文鼎宋体代替,不过效果可能较差。另一点是和西文不同的地方,这个设置允许我们指定粗体和斜体应用何种字体代替。之所以有这个区别,是因为中文不使用粗体表示强调,也不使用斜体表示引文或书名号。我一般会指定BoldFont 和 ItalicFont 为某种黑体,Windows 环境下是SimHei,Linux 下我会用文泉驿正黑避免版权问题。但具体操作上可能需要灵活调整。有些高校的论文模板,比如我母校就是明确要求强调段落必须使用楷体(SimKai),只能照着要求设置。
	\item \verb|\setCJKsansfont{ }| 和 \verb|\setCJKmonofont{ }|。这两个我没怎么用过,也说不清子丑寅卯来。一个比较安全的设置是统统设置成宋体,然后实际排版一次之后回头看看效果,再做必要的调整。我承认这样可能会比较难看,但至少不会有过于突兀的效果。
\end{itemize}
常用的本文字体配置:
\begin{lstlisting}
\setmainfont{CMU Serif},\setsansfont{CMU Sans Serif},\setmonofont{CMU Typewriter Text}
\setmainfont{TeX Gyre Pagella},\setsansfont{Droid Sans},\setmonofont{CMU Typewriter Text}
\setmainfont{TeX Gyre Terms},\setsansfont{Droid Sans},\setmonofont{CMU Typewriter Text}
\setmainfont{Linux Libertine O},\setsansfont{Droid Sans},\setmonofont{CMU Typewriter Text}
\end{lstlisting}

可能的话,\emph{不要使用 CJK 默认的中文粗体设置}。因为我们可能看到糟糕之极的效果,也就是所谓 Poor man's Bold “字体”。之所以带上引号,是因为它不是真的粗体,而是某种通过将原始宋体平移后得到的叠加效果。这种“字体”唯一的价值在于调试 LaTeX 文档时确保里面没有被误标记的粗体段落,绝对不能用来做出版用。


字号方面,提供了\verb+zihao=<-4|5|false>+来设置。

\CTeX 中提供了修改章节标题格式的接口,方法是使用\verb+heading=<true|false>+来打开或关闭,三个文档类中默认开启了上述接口。

\section{Beamer拾遗}
演讲时,时不时的总结,简单的重复演讲的主要信息。听众一般会在开头和结尾比较有注意力,总结是传递信息的“第二次机会”。

Beamer中默认使用无衬线字体,显示数学公式时不漂亮,可使用\verb|\usefonttheme[onlymath]{serif} |选择数学模式的字体。

\subsection{插入代码}
在Beamer中插入代码,应当在frame环境中加入\verb|[fragile]|选项再使用verbatim宏包。

\section{Tikz绘图}
Tikz是专业的矢量图工具,但学习成本比较高,我虽然看过文档,但并不熟练。替代的方式是,使用第三方矢量图软件(如Libre Office Draw或Dia)先做出图,再导出为\TeX 格式即可。

\subsection{思维导图}
使用以下代码可绘制出思维导图,如图\ref{fig:mindmap}所示:
\begin{lstlisting}
\documentclass[tikz]{standalone}
\usepackage{xcolor}
\usetikzlibrary{mindmap} %for mindmap
\definecolor{DeepSkyBlue4}{RGB}{0,104,139}
\begin{document}
\tikzstyle{level 2 concept}+=[sibling angle=40]
\begin{tikzpicture}[scale=0.49, transform shape]
\path[mindmap,concept color=black,text=white]
node[concept] {Pure Mathematics} [clockwise from=45]
child[concept color=DeepSkyBlue4]{
node[concept] {Analysis} [clockwise from=180]
child {
node[concept] {Multivariate \&amp; Vector Calculus}
[clockwise from=120]
child {node[concept] {ODEs}}}
child { node[concept] {Functional Analysis}}
child { node[concept] {Measure Theory}}
child { node[concept] {Calculus of Variations}}
child { node[concept] {Harmonic Analysis}}
child { node[concept] {Complex Analysis}}
child { node[concept] {Stochastic Analysis}}
child { node[concept] {Geometric Analysis}
[clockwise from=-40]
child {node[concept] {PDEs}}}}
child[concept color=black!50!green, grow=-40]{
node[concept] {Combinatorics} [clockwise from=10]
child {node[concept] {Enumerative}}
child {node[concept] {Extremal}}
child {node[concept] {Graph Theory}}}
child[concept color=black!25!red, grow=-90]{
node[concept] {Geometry} [clockwise from=-30]
child {node[concept] {Convex Geometry}}
child {node[concept] {Differential Geometry}}
child {node[concept] {Manifolds}}
child {node[concept,color=black!50!green!50!red,text=white] {Discrete Geometry}}
child {
node[concept] {Topology} [clockwise from=-150]
child {node [concept,color=black!25!red!50!brown,text=white]
{Algebraic Topology}}}}
child[concept color=brown,grow=140]{
node[concept] {Algebra} [counterclockwise from=70]
child {node[concept] {Elementary}}
child {node[concept] {Number Theory}}
child {node[concept] {Abstract} [clockwise from=180]
child {node[concept,color=red!25!brown,text=white] {Algebraic Geometry}}}
child {node[concept] {Linear}}}
node[extra concept,concept color=black] at (200:5) {Applied Mathematics}
child[grow=145,concept color=black!50!yellow] {
node[concept] {Probability} [clockwise from=180]
child {node[concept] {Stochastic Processes}}}
child[grow=175,concept color=black!50!yellow] {node[concept] {Statistics}}
child[grow=205,concept color=black!50!yellow] {node[concept] {Numerical Analysis}}
child[grow=235,concept color=black!50!yellow] {node[concept] {Symbolic Computation}};
\end{tikzpicture}
\end{document}
\end{lstlisting}

\begin{figure}[htpb]
	\centering
	\includegraphics[width=0.8\linewidth]{mindmap.png}
	\caption{使用Tikz制作思维导图示例。}\label{fig:mindmap}
\end{figure}

\section{文档类编写与个人文档订制}
我试图编写自己的文档类,但重复造轮子的效果并不比其他宏包好,因此放弃了学习文档类编写的努力。



很多网上的文档类是\verb|.dtx|格式,这是doc类文档生成的,执行同目录下\verb|latex test.ins|就会对|test.dtx|进行解析,得到相应的文档类文件,引用这个文档类即可进行编写。

虽然\CTeX 中自定义了常用的中文图表和章节格式,但有时可能需要自己重新定义,方法是:
\begin{lstlisting}
\renewcommand\contentsname{目录} 
\renewcommand\listfigurename{插图目录} 
\renewcommand\listtablename{表格目录} 
\renewcommand\refname{参考文献} 
\renewcommand\indexname{索引} 
\renewcommand\figurename{图} 
\renewcommand\tablename{表} 
\renewcommand\abstractname{摘要} 
\renewcommand\partname{部分} 
\renewcommand\appendixname{附录} 
\renewcommand\today{\number\year年\number\month月\number\day日} 
\end{lstlisting}


\section{总结}

\end{document}